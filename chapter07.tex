\chapter{Conclusion}

\section{Future work}

\subsection{Research data}

\subsection{Citation data}

Citation data, or bibliographic data of scholarly articles is a neuralgic point for the libraries. In the ``Western World'' and for large languages, the publishers are those players which traditionally built databases for the scholarly articles (such as Web of Knowledge, Scopus) instead of libraries. By and large there has been exceptions even in West European coutnries, and in case of smaller languages and for poorer countries the large publishers does not see the market value to publish scientific journals in vernacular languages therefore those journals are not covered in their databases. In the last two decades several different projects have been launched to make these metadata out of ``paywalls''. The largest of these project is the DOI database, but the larger part of DOI metadata is also not freely available, however the Initiative for Open Citations\footnote{\url{https://i4oc.org/}} works on making the citation data open. Recently WikiCite\footnote{\url{http://wikicite.org/}} is the largest freely available citation database based on the bibliographic data imported into Wikidata\footnote{\url{https://www.wikidata.org/wiki/Wikidata:Main_Page}}. It provides query interface and database dumps\footnote{\url{http://wikicite.org/access.html}}. Together with Jakob Voß, an volunteer of WikiCite and Wikidata we started a research project\footnote{Its code is en extension of the same codebase I wrote for the Europeana analysis. It is available at \url{https://github.com/pkiraly/metadata-qa-wikidata/wiki}.} to analyze the data. This research is in a preliminary stage. Now I highlight only one feature of the citation data namely \emph{page numbers}, which seems to be simple, but reveals some complex problems. Here I show three issues with page numbers. Wikidata uses a language neurtal notation for describing its semantic structure, the entities are denoted by `Q' and a number, while properties are denoted by `P' and a number. For example: P304 is the property of the page numbers. Its human readable label in English is ``page(s)''\footnote{\url{https://www.wikidata.org/wiki/Property:P304}}.

\subsubsection{1. Using article identifier as page number}

Example \#1. Q40154916\footnote{\url{https://www.wikidata.org/wiki/Q40154916}}: P304 = ``e0179574''

This article were published in PLoS ONE\footnote{\url{https://journals.plos.org/plosone/article?id=10.1371/journal.pone.0179574}}. The publisher provides citation text, and metadata in RIS and BibTeX format. The citation contains `e0179574', however it does not explain what exactly it means (as it neither explains any other elements):

Citation: Vincent WJB, Harvie EA, Sauer J-D, Huttenlocher A (2017) \emph{Neutrophil derived LTB4 induces macrophage aggregation in response to encapsulated Streptococcus iniae infection.} PLoS ONE 12(6): e0179574. \url{https://doi.org/10.1371/journal.pone.0179574}

In the PDF file\footnote{\url{https://journals.plos.org/plosone/article/file?id=10.1371/journal.pone.0179574&type=printable}} there are page numbers. It also does not explain what `e0179574' means. Metadata in RIS\footnote{https://journals.plos.org/plosone/article/citation/ris?id=10.1371/journal.pone.0179574} (excerpt) -- SP stands for starting page, EP stands for ending page. It is evident here, that `e0179574' is not a starting page. 
\begin{lstlisting}
SP  - e0179574
EP  - 
\end{lstlisting}

BibTeX\footnote{\url{https://journals.plos.org/plosone/article/citation/bibtex?id=10.1371/journal.pone.0179574}} contains page number, however does not contain article identifier:
\begin{lstlisting}
@article{10.1371/journal.pone.0179574,
  year = {2017},
  month = {06},
  volume = {12},
  pages = {1-16},
  ...
}
\end{lstlisting}

The DOI database follows the RIS metadata file instead of BibTeX\footnote{I used the following command to retrieve DOI metadata\\
\texttt{curl -H "Accept: application/rdf+xml" \\ http://data.crossref.org/10.1371/journal.pone.0179574}}:

\begin{lstlisting}
<rdf:RDF xmlns:rdf="http://www.w3.org/1999/02/22-rdf-syntax-ns#"
    xmlns:j.1="http://prismstandard.org/namespaces/basic/2.1/"
    xmlns:j.2="http://purl.org/ontology/bibo/" ...>
  <rdf:Description rdf:about=".../10.1371/journal.pone.0179574">
    <j.2:pageStart>e0179574</j.2:pageStart>
    <j.1:startingPage>e0179574</j.1:startingPage>
    ...
  </rdf:Description>
</rdf:RDF>
\end{lstlisting}

Example \#2. Q21820630\footnote{\url{https://www.wikidata.org/wiki/Q21820630}}: P304 = ``c181''

This paper were published in the British Medical Journal\footnote{\url{https://www.bmj.com/content/340/bmj.c181}}. It does not have a PDF version, the only online version is HTML.

Hrynaszkiewicz Iain, Norton Melissa L, Vickers Andrew J, Altman Douglas G. \emph{Preparing raw clinical data for publication: guidance for journal editors, authors, and peer reviewers} BMJ 2010; 340 :c181 \url{https://www.bmj.com/content/340/bmj.c181} (DOI: 10.1136/bmj.c181)

The RIS metadata\footnote{\url{https://www.bmj.com/highwire/citation/194003/ris}} contains bad page number, uses the article identifier:

\begin{lstlisting}
SP  - c181
\end{lstlisting}

In BibTex metadata\footnote{\url{https://www.bmj.com/highwire/citation/194003/bibtext}} there is no page number, but an `elocation-id` field is available:
\begin{lstlisting}
@article{Hrynaszkiewiczc181,
  elocation-id = {c181},
  ...
}
\end{lstlisting}

The DOI database follows the RIS metadata file instead of BibTeX\footnote{\url{http://data.crossref.org/10.1136/bmj.c181}}, however it repeats the same string as ending page:

\begin{lstlisting}
  <j.1:startingPage>c181</j.1:startingPage>
  <j.2:pageStart>c181</j.2:pageStart>
  <j.1:endingPage>c181</j.1:endingPage>
  <j.2:pageEnd>c181</j.2:pageEnd>
\end{lstlisting}

Conclusion: if there is a BibTeX source available, and it doesn't have page number element, but elocation-id, use that. 

JATS defines `<elocation-id>` element as ``replaces the start and end page elements just described for electronic-only publications;''\footnote{\url{https://jats.nlm.nih.gov/archiving/tag-library/1.1/element/elocation-id.html}}. (JATS is Journal Archiving and Interchange Tag Library. NISO JATS Version 1.1 (ANSI/NISO Z39.96-2015)).

In this journal (British Medical Journal) some article has a PDF version, (e.g. \url{https://doi.org/10.1136/bmj.d1584}, \url{https://doi.org/10.1136/bmj.e1454}, \url{https://doi.org/10.1136/bmj.a494}) where there are clearly page numbers. The journal provided RIS and BibTeX metadata does not contain page numbers in those cases.

\subsubsection{2. Wikidata contains extra info, which is not available elsewhere}

Q39877401\footnote{\url{https://www.wikidata.org/wiki/Q39877401}}: P304 = "108-17; quiz 118-9"

The article has been published in \emph{Orthopaedic Nursing}\footnote{\url{https://journals.lww.com/orthopaedicnursing/pages/articleviewer.aspx?year=2016&issue=03000&article=00010&type=abstract}}

Schroeder, Diana L.; Hoffman, Leslie A.; Fioravanti, Marie; Medley, Deborah Poskus; Zullo, Thomas G.; Tuite, Patricia K. \emph{Enhancing Nurses' Pain Assessment to Improve Patient Satisfaction} Orthopaedic Nursing: March/April 2016 - Volume 35 - Issue 2 - p 108–117. DOI: 10.1097/NOR.0000000000000226.

Page number in DOI\footnote{\url{http://data.crossref.org/10.1097/NOR.0000000000000226}} repeats the information provided at the journal:

\begin{lstlisting}
  <bibo:pageStart>108</bibo:pageStart>
  <prism:startingPage>108</prism:startingPage>

  <bibo:pageEnd>117</bibo:pageEnd>
  <prism:endingPage>117</prism:endingPage>
\end{lstlisting}

As we can see the publisher's citation the page number contains the first part of the page number string of the Wikidata value, but not the ``; quiz 118-9'' part. With some search on at the table of content we can find another article\footnote{\url{https://journals.lww.com/orthopaedicnursing/Citation/2016/03000/Enhancing_Nurses__Pain_Assessment_to_Improve.11.aspx#print-article-link}, DOI: 10.1097/NOR.0000000000000236} at page 118-119 of the same issue. There are no authors but the title is the same. It is categorized under ``CE Tests'' (where CE means continuing education). The two articles don't link directly to each other. They are different, and have different DOIs. The DOI database neither contains any link between them.

Wikidata should keep them separated, however it would require rather long time to investigate cases like this.

\subsubsection{3. Wikidata uses page number field to add comment}

Q28710224\footnote{\url{https://www.wikidata.org/wiki/Q28710224}}: P304 = "E3523; author reply E3524–5"

Alain Pierret, Valéry Zeitoun, Hubert Forestier: \emph{Irreconcilable differences between stratigraphy and direct dating cast doubts upon the status of Tam Pa Ling fossil.} Proceedings of the National Academy of Sciences Dec 2012, 109 (51) E3523; DOI: \url{http://doi.org/10.1073/PNAS.1216774109}, URL in journal: \url{https://www.pnas.org/content/109/51/E3523}, URL in Wikidata: \url{https://www.wikidata.org/wiki/Q28710224}.

E3524–5 is not part of the article. It is a related article, which is also available in Wikidata (as Q28710226\footnote{\url{https://www.wikidata.org/wiki/Q28710226}}):

Fabrice Demeter, Laura L. Shackelford, Kira E. Westaway, Philippe Duringer, Thongsa Sayavongkhamdy, and Anne-Marie Bacon: \emph{Reply to Pierret et al.: Stratigraphic and dating consistency reinforces the status of Tam Pa Ling fossil} PNAS December 18, 2012 109 (51) E3524-E3525; \url{https://doi.org/10.1073/pnas.1217629109}, URL in journal: \url{https://www.pnas.org/content/109/51/E3524}.

These two articles are not interlinked with distinct properties. We could suppose, that the occurrence `author reply' in other Wikidata records' page number could reveal similar hidden links.

\subsection{Fixing issues -- is that possible?}

The Swedish National Heritage Board just launched a project called \emph{Wikimedia Commons Data Roundtripping}\footnote{\url{https://outreach.m.wikimedia.org/wiki/GLAM/Newsletter/February_2019/Contents/Special_story}}. Roundtripping is the name of the workflow in which a cultural heritage institution publish their data in Wikimedia Commons, the users enrich these openly available media (such as adding translations of descriptive texts into other languages, identifying people, names and aliases, locations and subject matter or linking to authority data and using it to retrieve third party contributions from other memory organisations), then institutions ingest these data and update their original database.

Wikidata doesn't seem to be the platform where the data are generated, but rather it is a place where data created elsewhere are imported and maybe enriched. Additionally, the bibliographical data in general seems to have their own data flow: from different sources they are duplicated into different targets, such as DOI database(s), commercial discovery interfaces (Scopus, Web of Science, Google Scholar, Microsoft x (?), Springer Knowledge), institutional repositories, open data platforms (Wikidata, Wikipedia, DBPedia) etc. It is clear, that it would be impossible to fix the data in each platform, and probably the best place to fix them in their origin.

Who are responsible for metadata published in DOI database?

According to eRA's DOI experts Timo Gnadt and Sven Bingert DOI data is updated by the data providers, which are the institutions behind the landing page, thus in our case there are the publishers of the journals.

In one case I wrote a report to Springer Nature regarding to the page numbers for Wilkinson et al, \emph{The FAIR Guiding Principles for scientific data management and stewardship}. Scientific Data volume 3, Article number: 160018 (2016) \url{https://www.nature.com/articles/sdata201618}, but they did not reacted properly.

This project could provide Wikidata suggestions to fix the data in a format which could be used by bots to update the content. If such a fixing process is implemented, the Wikidata ingestion should be aware of the updated information, to be sure that they are not overwritten by a next ingestion process. Also these corrections should be sent to the publishers.