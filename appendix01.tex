\chapter{Metadata assessment bibliography}

\chapterauthor{Péter Király, Sarah Potvin, Nikos Palavitsinis, Anna Neatrour, Ayla Stein, Sara R., Juliane Stiller, Kevin Clair, Pru Mitchell, Corey Harper, Christina Harlow, Laura Akerman, Laura J. Smart, David Maus, jenyoung, Kate Flynn, leonardaa, Cassandra Baker}

This bibliography is the result of a community effort. It was built on Zotero platform as Metadata Assessment Group Library\footnote{\url{https://www.zotero.org/groups/488224/metadata_assessment}}. The bibliography was initialized by Corey Harper in February 2016, and DFL Metadata Assessment working group used for recording items found during their environmental scan. Soon Europeana Data Quality Committee also started contributing to it. During my PhD research I intensively used it as well. According to the Zotero API\footnote{\url{https://api.zotero.org/groups/488224/items}. In order to retrieve every items, one should have an API key, and should use parameters `start' and `limit' (see Zotero API documentation at \url{https://www.zotero.org/support/dev/web_api/v3/basics}.} the creators of the bibliography entries are Péter Király (94 items), Sarah Potvin (32), Nikos Palavitsinis (18), Anna Neatrour, Ayla Stein, Sara R. (7), Juliane Stiller, Kevin Clair (4), Pru Mitchell, Corey Harper, Christina Harlow, Laura Akerman, Laura J. Smart (2), David Maus, jenyoung, Kate Flynn, leonardaa. Cassandra Baker contributed with improving of existing bibliography items.

\begin{center}
    *
\end{center}

Acosta, Maribel, Amrapali Zaveri, Elena Simperl, Dimitris Kontokostas, Sören Auer, and Jens Lehmann. “Crowdsourcing Linked Data Quality Assessment.” In \emph{The Semantic Web – ISWC 2013}, 8219:260–76. Lecture Notes in Computer Science. Heidelberg: Springer, 2013. \url{https://doi.org/10.1007/978-3-642-41338-4_17}.

Albertoni, Riccardo, Monica De Martino, and Paola Podestà. “Quality Measures for Skos: ExactMatch Linksets: An Application to the Thesaurus Framework LusTRE.” \emph{Data Technologies and Applications} 2018 (n.d.). \url{https://doi.org/10.1108/DTA-05-2017-0037}.

Alemneh, Daniel Gelaw. “Metadata Quality Assessment: A Phased Approach to Ensuring Long-Term Access to Digital Resources.” \emph{Proceedings of the American Society for Information Science and Technology} 46, no. 1 (2009): 1–8. \url{https://doi.org/10.1002/meet.2009.1450460380}.

Askham, Nicola, Denise Cook, Martin Doyle, Helen Fereday, Mike Gibson, Ulrich Landbeck, Rob Lee, Chris Maynard, Gary Palmer, and Julian Schwarzenbach. “The Six Primary Dimensions For Data Quality Assessment. Defining Data Quality Dimensions.” DAMA UK, 2013. \url{https://www.whitepapers.em360tech.com/wp-content/files_mf/1407250286DAMAUKDQDimensionsWhitePaperR37.pdf}.

Bade, David. “The Perfect Bibliographic Record: Platonic Ideal, Rhetorical Strategy or Nonsense?” \emph{Cataloging \& Classification Quarterly} 46, no. 1 (2008): 109–33. \url{https://doi.org/10.1080/01639370802183081}.

Barton, Jane, Sarah Currier, and Jessie M. N. Hey. “Building Quality Assurance into Metadata Creation: An Analysis Based on the Learning Objects and e-Prints Communities of Practice.” In \emph{Papers and Project Reports for DC-2003 in Seattle, 28 September - 2 October 2003. Supporting Communities of Discourse and Practice}, 39–48, 2003. \url{http://dcpapers.dublincore.org/pubs/article/view/732}.

Beall, Jeffrey. “Metadata and Data Quality Problems in the Digital Library.” \emph{Journal of Digital Information} 6, no. 3 (2005): 20.

Bellini, Emanuele, and Paolo Nesi. “Metadata Quality Assessment Tool for Open Access Cultural Heritage Institutional Repositories.” In \emph{Information Technologies for Performing Arts, Media Access, and Entertainment}, 7990:90–103. Lecture Notes in Computer Science. Heidelberg: Springer, 2013. \url{https://doi.org/10.1007/978-3-642-40050-6_9}.

Bellini, Pierfrancesco, Ivan Bruno, Paolo Nesi, and Michela Paolucci. “IPR Centered Institutional Service and Tools for Content and Metadata Management.” \emph{International Journal of Software Engineering and Knowledge Engineering} 25, no. 08 (2015): 1237–70. \url{https://doi.org/10.1142/S0218194015500242}.

Bruce, Thomas R., and Diane I. Hillmann. “Metadata Quality in a Linked Data Context,” January 24, 2013. \url{https://blog.law.cornell.edu/voxpop/2013/01/24/metadata-quality-in-a-linked-data-context}.

———. “The Continuum of Metadata Quality: Defining, Expressing, Exploiting.” In \emph{Metadata in Practice}, edited by D. Hillman and E. Westbrooks, 238–56. Chicago, IL: ALA Editions, 2004. \url{http://ecommons.cornell.edu/handle/1813/7895}.

Cai, Li, and Yangyong Zhu. “The Challenges of Data Quality and Data Quality Assessment in the Big Data Era.” \emph{Data Science Journal} 14, no. 2 (2015). \url{https://doi.org/10.5334/dsj-2015-002}.

Cechinel, Cristian, Sandro da Silva Camargo, Salvador Sánchez-Alonso, and Miguel-Ángel Sicilia. “On the Search for Intrinsic Quality Metrics of Learning Objects.” In \emph{Metadata and Semantics Research}, 343 CCIS:49–60. Communications in Computer and Information Science. Berlin, Heidelberg: Springer, 2012. \url{https://doi.org/10/gfphbr}.

Cechinel, Cristian, Sandro da Silva Camargo, Miguel-Ángel Sicilia, and Salvador Sánchez-Alonso. “Mining Models for Automated Quality Assessment of Learning Objects.” \emph{Journal of Universal Computer Science} 22, no. 1 (2016): 94–113. \url{https://doi.org/10.3217/jucs-022-01-0094}.

Charbonneau, Mechael. “Production Benchmarks for Catalogers in Academic Libraries: Are We There Yet?” \emph{Library Resources and Technical Services} 49, no. 1 (January 2005): 40–48.

Charles, Valentine, Juliane Stiller, Péter Király, Wener Bailer, and Nuno Freire. “Data Quality Assessment in Europeana: Metrics for Multilinguality.” In \emph{Joint Proceedings of the 1st Workshop on Temporal Dynamics in Digital Libraries (TDDL 2017), the (Meta)-Data Quality Workshop (MDQual 2017) and the Workshop on Modeling Societal Future (Futurity 2017) (TDDL\_MDQual\_Futurity 2017) Co-Located with 21st International Conference on Theory and Practice of Digital Libraries (TPLD 2017), Grand Hotel Palace, Thessaloniki, Greece, 21 September 2017}, Vol. 2038. CEUR Workshop Proceedings. CEUR Workshop Proceedings, 2018. \url{http://ceur-ws.org/Vol-2038/paper6.pdf}.

Chen, Hung-Hsuan, Jian Wu, and C. Lee Giles. “Compiling Keyphrase Candidates for Scientific Literature Based on Wikipedia.” In \emph{Joint Proceedings of the 1st Workshop on Temporal Dynamics in Digital Libraries (TDDL 2017), the (Meta)-Data Quality Workshop (MDQual 2017) and the Workshop on Modeling Societal Future (Futurity 2017) (TDDL\_MDQual\_Futurity 2017) Co-Located with 21st International Conference on Theory and Practice of Digital Libraries (TPLD 2017), Grand Hotel Palace, Thessaloniki, Greece, 21 September 2017}, Vol. 2038. CEUR Workshop Proceedings. CEUR Workshop Proceedings, 2018. \url{http://ceur-ws.org/Vol-2038/paper4.pdf}.

Chen, Ya-Ning, Chun-Ya Wen, Hui-Pin Chen, Yen-Hung Lin, and Hon-Chung Sum. “Metrics for Metadata Quality Assurance and Their Implications for Digital Libraries.” In \emph{Digital Libraries: For Cultural Heritage, Knowledge Dissemination, and Future Creation (Proceedings of the 13th International Conference on Asia-Pacific Digital Libraries, 24–27 October 2011, Beijing, China.)}, 138–47. Beijing, China: Springer, 2011. \url{https://doi.org/10.1007/978-3-642-24826-9_19}.

Clair, Kevin. “Technical Debt as an Indicator of Library Metadata Quality.” \emph{D-Lib Magazine} 22, no. 11/12 (December 2016). \url{https://doi.org/10.1045/november2016-clair}.

Clements, Kati, Jan Pawlowski, and Nikos Manouselis. “Open Educational Resources Repositories Literature Review – Towards a Comprehensive Quality Approaches Framework.” \emph{Computers in Human Behavior}, Computers in Human Behavior, 51, Part B (2015): 1098–1106. \url{https://doi.org/10.1016/j.chb.2015.03.026}.

———. “Why Open Educational Resources Repositories Fail - Review of Quality Assurance Approaches.” In \emph{EDULEARN14 Proceedings. 6th International Conference on Education and New Learning Technologies Barcelona, Spain, 2014}, 929–39. Barcelona, Spain: IATED, International Association of Technology, Education and Development, 2014. \url{https://jyx.jyu.fi/dspace/handle/123456789/44031}.

Conrad, Suzanna. “Using Google Tag Manager and Google Analytics to Track DSpace Metadata Fields as Custom Dimensions.” \emph{The Code4Lib Journal}, no. 27 (January 21, 2015). \url{http://journal.code4lib.org/articles/10311}.

Debattista, Jeremy, Sören Auer, and Christoph Lange. “Luzzu - A Methodology and Framework for Linked Data Quality Assessment.” \emph{Journal of Data and Information Quality} 8, no. 1 (2016): 4:1-4:32. \url{https://doi.org/10.1145/2992786}.

Debattista, Jeremy, Christoph Lange, and Sören Auer. “Representing Dataset Quality Metadata Using Multi-Dimensional Views.” In \emph{SEM ’14. Proceedings of the 10th International Conference on Semantic Systems}, 92–99. Leipzig, Germany: ACM, 2014. \url{https://doi.org/10.1145/2660517.2660525}.

Decourselle, Joffrey, Fabien Duchateau, Trond Aalberg, Naimdjon Takhirov, and Nicolas Lumineau. “BIB-R: A Benchmark for the Interpretation of Bibliographic Records.” In \emph{Research and Advanced Technology for Digital Libraries}, edited by Norbert Fuhr, László Kovács, Thomas Risse, and Wolfgang Nejdl, 9819:163–174. Lecture Notes in Computer Science. Hannover, Germany: Springer International Publishing, 2016. \url{https://doi.org/10.1007/978-3-319-43997-6}.

———. “Open Datasets for Evaluating the Interpretation of Bibliographic Records.” In \emph{Proceedings of the 16th ACM/IEEE-CS on Joint Conference on Digital Libraries}, 253–54. Newark, New Jersey, USA: ACM, 2016. \url{https://doi.org/10.1145/2910896.2925457}.

Degerstedt, Stina, and Joakim Philipson. “Lessons Learned from the First Year of E-Legal Deposit in Sweden: Ensuring Metadata Quality in an Ever-Changing Environment.” \emph{Cataloging \& Classification Quarterly} 54, no. 7 (2016): 468–82. \url{https://doi.org/10.1080/01639374.2016.1197170}.

Diakopoulos, Nicholas, Sorelle Friedler, Marcelo Arenas, Solon Barocas, Michael Hay, Bill Howe, H. V. Jagadish, et al. “Principles for Accountable Algorithms and a Social Impact Statement for Algorithms.” FAT/ML (Fairness, Accountability, and Transparency in Machine Learning), July 2016. \url{http://www.fatml.org/resources/principles-for-accountable-algorithms}.

DLF Aquifer Metadata Working Group, and Digital Library Federation. “Digital Library Federation / Aquifer Implementation Guidelines for Shareable MODS Records, Version 1.1.” Standard/Guidelines. Digital Library Federation, November 2009. \url{https://wiki.dlib.indiana.edu/display/DLFAquifer/DLF+Aquifer+Public+Metadata+Documents?preview=/28330/120160257/DLFMODS_ImplementationGuidelines.pdf}.

DLF/NSDL Working Group on OAI PMH Best Practices. “Best Practices for OAI PMH DataProvider Implementations and Shareable Metadata.” Washington, D.C.: Digital Library Federation, 2007. \url{https://old.diglib.org/pubs/dlf108.pdf}.

Dodds, Leigh. “Quality Indicators for Linked Data Datasets,” n.d. \url{http://answers.semanticweb.com/questions/1072/quality-indicators-for-linked-data-datasets}.

Doorn, Peter, and Eleftheria Tsoupra. “A Simple Approach to Assessing the FAIRness of Data in Trusted Digital Repositories.” In \emph{Joint Proceedings of the 1st Workshop on Temporal Dynamics in Digital Libraries (TDDL 2017), the (Meta)-Data Quality Workshop (MDQual 2017) and the Workshop on Modeling Societal Future (Futurity 2017) (TDDL\_MDQual\_Futurity 2017) Co-Located with 21st International Conference on Theory and Practice of Digital Libraries (TPLD 2017), Grand Hotel Palace, Thessaloniki, Greece, 21 September 2017}, Vol. 2038. CEUR Workshop Proceedings. CEUR Workshop Proceedings, 2018. \url{http://ceur-ws.org/Vol-2038/invited2.pdf}.

DPLAFest Participants. “Metadata Quality Research,” 2015. \url{https://docs.google.com/document/d/15pmA276_fxShkCEagoloJwCXH89PhrF3qWBgB8xSrag/edit#}.

Dublin Core Metadata Initiative. “DCMI Task Group RDF Application Profiles,” 2014. \url{http://wiki.dublincore.org/index.php/RDF_Application_Profiles}.

Durco, Matej, and Menzo Windhouwer. “The CMD Cloud.” In \emph{Proceedings of LREC 2014}, edited by N. Calzolari, K. Choukri, T. Declerck, H. Loftsson, B. Maegaard, and J. Mariani, 687–90. Reykjavik, Iceland, 2014. \url{http://www.lrec-conf.org/proceedings/lrec2014/pdf/156_Paper.pdf}.

Dushay, Naomi, and Diane I. Hillmann. “Analyzing Metadata for Effective Use and Re-Use.” In \emph{Proceedings, Dublin Core Metadata Conference, DC-2003}. Seattle, Washington, USA: Dublin Core Metadata Initiative, 2003. \url{http://dcpapers.dublincore.org/pubs/article/view/744}.

Efron, Miles. “Metadata Use in OAI-Compliant Institutional Repositories.” \emph{Journal of Digital Information} 8, no. 2 (2007). \url{http://people.ischool.illinois.edu/~mefron/papers/efron-metadatause.pdf}.

Ellefi, Mohamed Ben, Zohra Bellahsene, John Breslin, Elena Demidova, Stefan Dietze, Julian Szymanski, and Konstantin Todorov. “RDF Dataset Profiling - a Survey of Features, Methods, Vocabularies and Applications.” \emph{Semantic Web} Preprint, no. Preprint (2017). \url{http://www.semantic-web-journal.net/content/rdf-dataset-profiling-survey-features-methods-vocabularies-and-applications}.

Europeana Tech. “Evaluation and Enrichments Task Report Outcomes,” 2015. \url{http://pro.europeana.eu/get-involved/europeana-tech/europeanatech-task-forces/evaluation-and-enrichments}.

Färber, Michael, Frederic Bartscherer, Carsten Menne, and Achim Rettinger. “Linked Data Quality of DBpedia, Freebase, OpenCyc, Wikidata, and YAGO.” \emph{Semantic Web} Preprint, no. Preprint (2017): 1–53. \url{https://doi.org/10.3233/SW-170275}.

Fischer, Karen S. “Critical Views of LCSH, 1990–2001: The Third Bibliographic Essay.” \emph{Cataloging \& Classification Quarterly} 41, no. 1 (October 1, 2005): 63–109. \url{https://doi.org/10.1300/J104v41n01_05}.

Foulonneau, Muriel. “Information Redundancy across Metadata Collections.” \emph{Information Processing \& Management} 43, no. 3 (2007): 740–51. \url{http://dx.doi.org/10.1016/j.ipm.2006.06.004}.

Foulonneau, Muriel, and Timothy W. Cole. “Strategies for Reprocessing Aggregated Metadata.” In \emph{Research and Advanced Technology for Digital Libraries}, edited by Andreas Rauber, Stavros Christodoulakis, and A. Min Tjoa, 290–301. Berlin, Heidelberg: Springer Berlin Heidelberg, 2005. \url{https://www.researchgate.net/profile/Muriel_Foulonneau/publication/221176072_Strategies_for_Reprocessing_Aggregated_Metadata/links/54173a450cf2f48c74a403d1.pdf}.

Foulonneau, Muriel, and Jenn Riley. \emph{Metadata for Digital Resources: Implementation, Systems Design and Interoperability}. Oxford: Chandos Publishing, 2008.

Fowler, Dan, Jo Barratt, and Paul Walsh. “Frictionless Data: Making Research Data Quality Visible.” \emph{International Journal of Digital Curation} 12, no. 2 (May 13, 2018): 274–85. \url{https://doi.org/10.2218/ijdc.v12i2.577}.

Friesen, Norm. “International LOM Survey: Report (Draft),” June 27, 2004. \url{http://arizona.openrepository.com/arizona/bitstream/10150/106473/1/LOM_Survey_Report2.doc}.

Fürber, Christian, and Martin Hepp. “Towards a Vocabulary for Data Quality Management in Semantic Web Architectures.” presented at the First International Workshop on Linked Web Data Management, Uppsala, Sweden, 2011. \url{http://www.slideshare.net/cfuerber/towards-a-vocabulary-for-data-quality-management-in-semantic-web-architectures}.

Gavrilis, Dimitris, Dimitra-Nefeli Makri, Leonidas Papachristopoulos, Stavros Angelis, Konstantinos Kravvaritis, Christos Papatheodorou, and Panos Constantopoulos. “Measuring Quality in Metadata Repositories.” In \emph{Proceedings from the 19th International Conference on Theory and Practice of Digital Libraries}, edited by S. Kapidakis, C. Mazurek, and M. Werla, 56–67. Poznań, Poland: Springer, 2015. \url{https://doi.org/10.1007/978-3-319-24592-8_5}.

“Gendered Expectations for Leadership in Libraries – In the Library with the Lead Pipe.” Accessed May 9, 2017. \url{http://www.inthelibrarywiththeleadpipe.org/2015/libleadgender/}.

GO FAIR Metrics Group. “FAIR Metrics.” Accessed April 18, 2019. \url{http://fairmetrics.org/}.

Gonçalves, Marcos André, Bárbara L. Moreira, Edward A. Fox, and Layne T. Watson. “‘What Is a Good Digital Library?’ – A Quality Model for Digital Libraries.” \emph{Information Processing \& Management} 43, no. 5 (June 3, 2007): 1416–37. \url{https://doi.org/10.1016/j.ipm.2006.11.010}.

Goovaerts, Marc, and Dirk Leinders. “Metadata Quality Evaluation of a Repository Based on a Sample Technique.” In \emph{Metadata and Semantics Research}, 343 CCIS:181–89. Communications in Computer and Information Science. Berlin, Heidelberg: Springer, 2012. \url{https://doi.org/10/gfphbs}.

Greenberg, Jane, Kristina Spurgin, and Abe Crystal. “Final Report for the AMeGA (Automatic Metadata Generation Applications) Project,” February 17, 2005. \url{http://www.loc.gov/catdir/bibcontrol/lc_amega_final_report.pdf}.

Groskopf, Christopher. “The Quartz Guide to Bad Data.” Quartz, 2015. \url{https://qz.com/572338/the-quartz-guide-to-bad-data/}.

Gueguen, Gretchen. “Metadata Quality at Scale: Metadata Quality Control at the Digital Public Library of America.” \emph{Journal of Digital Media Management} 7, no. 2 (2019): 115–26.

Guinchard, Carolyn. “Dublin Core Use in Libraries: A Survey.” \emph{OCLC Systems \& Services: International Digital Library Perspectives} 18, no. 1 (2006): 40–50. \url{https://doi.org/10.1108/10650750210418190}.

Guy, Marieke, Andy Powell, and Michael Day. “Improving the Quality of Metadata in Eprint Archives.” \emph{Ariadne}, no. 38 (2004). \url{http://www.ariadne.ac.uk/issue38/guy/}.

Harper, Corey. “Metadata Analytics, Visualization, and Optimization: Experiments in Statistical Analysis of the Digital Public Library of America (DPLA).” \emph{The Code4Lib Journal}, no. 33 (July 19, 2016). \url{http://journal.code4lib.org/articles/11752}.

Haslhofer, Bernard, and Wolfgang Klas. “A Survey of Techniques for Achieving Metadata Interoperability.” \emph{ACM Computing Surveys} 42, no. 2 (February 2010): 1–37. \url{https://doi.org/10.1145/1667062.1667064}.

Hillmann, Diane I. “Metadata Quality: From Evaluation to Augmentation.” \emph{Cataloging \& Classification Quarterly} 46, no. 1 (2008): 65–80. \url{https://doi.org/10.1080/01639370802183008}.

Hillmann, Diane I., Naomi Dushay, and Jon Phipps. “Improving Metadata Quality: Augmentation and Recombination,” 2004. \url{http://www.cs.cornell.edu/naomi/DC2004/MetadataAugmentation--DC2004.pdf}.

Höffernig, Martin, Thomas Orgel, Silvia Russegger, and Werner Bailer. “Assessing Quality in Automated Metadata Aggregation and Mapping Services,” 1–6. Poznań, Poland: EEXCEES, 2015. \url{http://eexcess.eu/wp-content/uploads/2013/03/JoanneumResearch_Assessing_Quality.pdf}.

———. “Assessing Quality in Automated Metadata Aggregation and Mapping Services,” n.d. \url{http://eexcess.eu/wp-content/uploads/2013/03/JoanneumResearch_Assessing_Quality.pdf}.

Hooland, Seth van. “Metadata Quality in the Cultural Heritage Sector: Stakes, Problems and Solutions.” PhD, Université Libre de Bruxelles, Faculté de Philosophie et Letteres, 2009. \url{http://homepages.ulb.ac.be/~svhoolan/these.pdf}.

Huang, Yu, and Fei Chiang. “Refining Duplicate Detection for Improved Data Quality.” In \emph{Joint Proceedings of the 1st Workshop on Temporal Dynamics in Digital Libraries (TDDL 2017), the (Meta)-Data Quality Workshop (MDQual 2017) and the Workshop on Modeling Societal Future (Futurity 2017) (TDDL\_MDQual\_Futurity 2017) Co-Located with 21st International Conference on Theory and Practice of Digital Libraries (TPLD 2017), Grand Hotel Palace, Thessaloniki, Greece, 21 September 2017}, Vol. 2038. CEUR Workshop Proceedings. CEUR Workshop Proceedings, 2018. \url{http://ceur-ws.org/Vol-2038/paper3.pdf}.

Hughes, Baden. “Metadata Quality Evaluation: Experience from the Open Language Archives Community.” In \emph{Digital Libraries: International Collaboration and Cross-Fertilization}, 320–29. Lecture Notes in Computer Science. Berlin, Heidelberg: Springer, 2004. \url{https://doi.org/10.1007/978-3-540-30544-6_34}.

Jackson, Amy, Myung-Ja Han, Kurt Groetsch, Megan Mustafoff, and Timothy W. Cole. “Dublin Core Metadata Harvested Through OAI-PMH.” \emph{Journal of Library Metadata} 8, no. 1 (2008): 5–21. \url{https://doi.org/10.1080/10911360802076682}.

Jay, Michael, Betsy Simpson, and Doug Smith. “CatQC and Shelf-Ready Material: Speeding Collections to Users While Preserving Data Quality.” \emph{Information Technology and Libraries} 28, no. 1 (2009): 41–48. \url{https://doi.org/10.6017/ital.v28i1.3171}.

Kaffee, Lucie-Aimée, Alessandro Piscopo, Pavlos Vougiouklis, Elena Simperl, Leslie Carr, and Lydia Pintscher. “A Glimpse into Babel: An Analysis of Multilinguality in Wikidata.” In \emph{OpenSym ’17 Proceedings of the 13th International Symposium on Open Collaboration}, 14:1--14:5. OpenSym ’17. ACM, 2017. \url{https://doi.org/10.1145/3125433.3125465}.

Kaffee, Lucie-Aimée, and Elena Simperl. “The Human Face of the Web of Data: A Cross-Sectional Study of Labels,” 137:66–77. Procedia Computer Science, 2018. \url{https://doi.org/10.1016/j.procs.2018.09.007}.

Kapidakis, Sarantos. “Comparing Metadata Quality in the Europeana Context.” In \emph{PETRA ’12 Proceedings of the 5th International Conference on PErvasive Technologies Related to Assistive Environments. Heraklion, Crete, Greece — June 06 - 08, 2012}, 25:1-25:8. Heraklion, Crete, Greece: ACM, 2012. \url{https://doi.org/10.1145/2413097.2413129}.

———. “Rating Quality in Metadata Harvesting.” In \emph{PETRA ’15 Proceedings of the 8th ACM International Conference on PErvasive Technologies Related to Assistive Environments. Article No. 65. Corfu, Greece — July 01 - 03, 2015.}, 65:1-65:8. Corfu, Greece: ACM, 2015. \url{https://doi.org/10.1145/2769493.2769512}.

Kelly, Brian, Amanda Closier, and Debra Hiom. “Gateway Standardization: A Quality Assurance Framework for Metadata.” \emph{Library Trends} 53, no. 4 (Spring 2005): 637–50.

Király, Péter. “A Metadata Quality Assurance Framework,” September 2015. \url{http://pkiraly.github.io/metadata-quality-project-plan.pdf}.

———. “Metadata Quality Assurance Framework,” 2015. \url{http://pkiraly.github.io/}.

———. “Towards an Extensible Measurement of Metadata Quality.” In \emph{DATeCH2017: Proceedings of the 2nd International Conference on Digital Access to Textual Cultural Heritage}, 111–115. New York, NY, USA: ACM, 2017. \url{https://doi.org/10.1145/3078081.3078109}.

Király, Péter, and Marco Büchler. “Measuring Completeness as Metadata Quality Metric in Europeana.” In \emph{2018 IEEE International Conference on Big Data (Big Data)}, 2711–2720. IEEE, 2018. \url{https://doi.org/10.1109/BigData.2018.8622487}.

Király, Péter, Juliane Stiller, Valentine Charles, Werner Bailer, and Nuno Freire. “Evaluating Data Quality in Europeana: Metrics for Multilinguality.” In \emph{Metadata and Semantic Research}, edited by Emmanouel Garoufallou, Fabio Sartori, Rania Siatri, and Marios Zervas, 846:199–211. Communications in Computer and Information Science. Cham: Springer International Publishing, 2019. \url{https://doi.org/10.1007/978-3-030-14401-2_19}.

Kontokostas, Dimitris, Christian Mader, Christian Dirschl, Katja Eck, Michael Leuthold, Jens Lehmann, and Sebastian Hellmann. “Semantically Enhanced Quality Assurance in the JURION Business Use Case.” In \emph{The Semantic Web. Latest Advances and New Domains. ESWC 2016}, 9678:661–76. Lecture Notes in Computer Science. Heidelberg: Springer, 2016. \url{https://doi.org/10.1007/978-3-319-34129-3_40}.

Kontokostas, Dimitris, Amrapali Zaveri, Sören Auer, and Jens Lehmann. “TripleCheckMate: A Tool for Crowdsourcing the Quality Assessment of Linked Data,” 394:265–72. Communications in Computer and Information Science. Heidelberg: Springer, 2013. \url{https://doi.org/10.1007/978-3-642-41360-5_22}.

Koster, Lukas. “Analysing Library Data Flows for Efficient Innovation.” \emph{Commonplace.Net} (blog), November 27, 2014. \url{http://commonplace.net/2014/11/library-data-flows/}.

Lagoze, Carl, Dean B. Krafft, Susan Jesuroga, Tim Cornwell, Ellen J. Cramer, and Eddie Shin. “An Information Network Overlay Architecture for the NSDL.” In \emph{JCDL ’05 Proceedings of the 5th ACM/IEEE-CS Joint Conference on Digital Libraries}, 384–384. New York: ACM, 2005. \url{https://doi.org/10.1145/1065385.1065487}.

Lei, Yuangui, Marta Sabou, Vanessa López, Jianhan Zhu, Victoria S. Uren, and Enrico Motta. “An Infrastructure for Acquiring High Quality Semantic Metadata.” In \emph{The Semantic Web: Research and Applications}, 4011:230–44. Lecture Notes in Computer Science. Berlin Heidelberg: Springer, 2006. \url{https://doi.org/10.1007/11762256_19}.

Lim, Shirley, and Chern Li Liew. “Metadata Quality and Interoperability of GLAM Digital Images.” \emph{Aslib Proceedings} 63, no. 5 (September 20, 2011): 484–98. \url{https://doi.org/10.1108/00012531111164978}.

Lopatin, Laurie. “Metadata Practices in Academic and Non-Academic Libraries for Digital Projects: A Survey.” \emph{Cataloging \& Classification Quarterly} 48, no. 8 (n.d.): 716–42. \url{http://dx.doi.org/10.1080/01639374.2010.509029}.

Loshin, David. “Building a Data Quality Scorecard for Operational Data Governance.” SAS Institute Inc., 2013. \url{http://www.sas.com/content/dam/SAS/en_us/doc/whitepaper1/building-data-quality-scorecard-for-operational-data-governance-106025.pdf}.

Lovins, Daniel, and Diane I. Hillmann. “Broken-World Vocabularies.” \emph{D-Lib Magazine} 23, no. 3/4 (March/April) (2017). \url{https://doi.org/10.1045/march2017-lovins}.

Ma, Shanshan, Caimei Lu, Xia Lin, and Mike Galloway. “Evaluating the Metadata Quality of the IPL.” \emph{Proceedings of the American Society for Information Science and Technology} 46, no. 1 (January 1, 2009): 1–17. \url{https://doi.org/10.1002/meet.2009.1450460249}.

Mader, Christian, Bernard Haslhofer, and Antoine Isaac. “Finding Quality Issues in SKOS Vocabularies.” In \emph{Theory and Practice of Digital Libraries: Second International Conference, TPDL 2012, Paphos, Cyprus, September 23-27, 2012. Proceedings}, 7489:222–33. Lecture Notes in Computer Science. Heidelberg: Springer, 2012. \url{https://doi.org/10.1007/978-3-642-33290-6_25}.

Madnick, Stuart E., Richard Y. Wang, Yang W. Lee, and Hongwei Zhu. “Overview and Framework for Data and Information Quality Research.” \emph{ACM Journal of Data and Information Quality} 1, no. 1 (2009): 1–22. \url{https://doi.org/10.1145/1515693.1516680}.

Margaritopoulos, Merkourios, Thomas Margaritopoulos, Ioannis Mavridis, and Athanasios Manitsaris. “Quantifying and Measuring Metadata Completeness.” \emph{Journal of the American Society for Information Science and Technology} 63, no. 4 (April 2012): 724–37. \url{https://doi.org/10.1002/asi.21706}.

Margaritopoulos, Thomas, Merkourios Margaritopoulos, Ioannis Mavridis, and Athanasios Manitsaris. “A Conceptual Framework for Metadata Quality Assessment.” Berlin, Germany, 2008. \url{http://dcpapers.dublincore.org/pubs/article/view/923}.

———. “A Fine-Grained Metric System for the Completeness of Metadata.” In \emph{Conference Paper in Communications in Computer and Information Science}, edited by F. Sartori, M. Á. Sicilia, and N. Manouselis, 83–94. Milan, Italy: Springer, 2009. \url{https://doi.org/10.1007/978-3-642-04590-5_8}.

Matienzo, Mark A., and Amy Rudersdorf. “The Digital Public Library of America Ingestion Ecosystem: Lessons Learned After One Year of Large-Scale Collaborative Metadata Aggregation,” 12–23. Austin, Texas, 2014. \url{http://dcpapers.dublincore.org/pubs/article/view/3700}.

Micic, Natasha, Daniel Neagu, I Felician Campean, and Esmaeil Habib Zadeh. “Towards a Data Quality Framework for Heterogeneous Data,” 2017. \url{https://bradscholars.brad.ac.uk/bitstream/handle/10454/12323/PID4808071.pdf?sequence=3\&isAllowed=y}.

“MODS Guidelines Levels of Adoption.” Wiki. MODS Guidelines Levels of Adoption - American Social History Online, June 30, 2009. \url{https://wiki.dlib.indiana.edu/display/DLFAquifer/MODS+Guidelines+Levels+of+Adoption}.

Moen, William E., Erin L. Stewart, and Charles R. McClure. “The Role of Content Analysis in Evaluating Metadata for the U.S. Government Information Locator Service (GILS): Results from an Exploratory Study Citations, Rights, Re-Use.” University of North Texas Libraries, Digital Library, 1997. \url{https://digital.library.unt.edu/ark:/67531/metadc36312/}.

Najjar, Jehad, and Erik Duval. “Actual Use of Learning Objects and Metadata: An Empirical Analysis.” \emph{TCDL Bulletin} 2, no. 2 (2006). \url{http://www.ieee-tcdl.org/Bulletin/v2n2/najjar/najjar.html}.

Najjar, Jehad, Stefaan Ternier, and Erik Duval. “The Actual Use of Metadata in Ariadne: An Empirical Analysis.” In \emph{Proc. ARIADNE 3rd International Conference (2003)}, 6, 2003. \url{http://citeseerx.ist.psu.edu/viewdoc/download?doi=10.1.1.93.3666\&rep=rep1\&type=pdf}.

Neto, Ciro Baron, Dimitris Kontokostas, Sebastian Hellmann, Kay Müller, and Martin Brümmer. “Assessing Quantity and Quality of Links Between Linked Data Datasets.” In \emph{Proceedings of the Workshop on Linked Data on the Web Co-Located with the 25th International World Wide Web Conference (WWW 2016)}, edited by Sören Auer, Tim Berners-Lee, Christian Bizer, and Tom Heath, Vol. 1593. CEUR Workshop Proceedings. Montreal, Canada: CEUR Workshop Proceedings, 2016. \url{http://ceur-ws.org/Vol-1593/#article-07}.

Newman, David, Kat Hagedorn, Chaitanya Chemudugunta, and Padhraic Smyth. “Subject Metadata Enrichment Using Statistical Topic Models.” In \emph{Proceedings of the 7th ACM/IEEE-CS Joint Conference on Digital Libraries}, 366–375. JCDL ’07. New York, NY, USA: ACM, 2007. \url{https://doi.org/10.1145/1255175.1255248}.

Ngomo, Axel-Cyrille Ngonga, Sören Auer, Jens Lehmann, and Amrapali Zaveri. “Introduction to Linked Data and Its Lifecycle on the Web.” In \emph{Reasoning Web. Reasoning on the Web in the Big Data Era: 10th International Summer School 2014, Athens, Greece, September 8-13, 2014. Proceedings}, 1–99. Heidelberg: Springer, 2014. \url{http://jens-lehmann.org/files/2013/reasoning_web_linked_data.pdf}.

Nichols, David M., Dana McKay, and Michael B. Twidale. “A Lightweight Metadata Quality Tool.” In \emph{Proceedings of the 8th ACM/IEEE-CS Joint Conference on Digital Libraries - JCDL ’08}, 385–88. New York: ACM Press, 2008. \url{https://doi.org/10.1145/1378889.1378957}.

Noh, Younghee. “A Study on Metadata Elements for Web-Based Reference Resources System Developed through Usability Testing.” \emph{Library Hi Tech} 29, no. 2 (2011): 242–65. \url{https://doi.org/10.1108/07378831111138161}.

Nomikos, Vangelis. “Repolytics: Identifying Measurable Insights for Digital Repositories.” In \emph{Joint Proceedings of the 1st Workshop on Temporal Dynamics in Digital Libraries (TDDL 2017), the (Meta)-Data Quality Workshop (MDQual 2017) and the Workshop on Modeling Societal Future (Futurity 2017) (TDDL\_MDQual\_Futurity 2017) Co-Located with 21st International Conference on Theory and Practice of Digital Libraries (TPLD 2017), Grand Hotel Palace, Thessaloniki, Greece, 21 September 2017}, Vol. 2038. CEUR Workshop Proceedings. CEUR Workshop Proceedings, 2018. \url{http://ceur-ws.org/Vol-2038/paper7.pdf}.

Nwala, Alexander C., Michele C. Weigle, and Michael L. Nelson. “Bootstrapping Web Archive Collections from Social Media.” In \emph{Proceedings of the 29th on Hypertext and Social Media}, 64–72. HT ’18. Baltimore, MD, USA: ACM, 2018. \url{https://doi.org/10.1145/3209542.3209560}.

Ochoa, Xavier, and Erik Duval. “Automatic Evaluation of Metadata Quality in Digital Repositories.” \emph{International Journal on Digital Libraries} 10, no. 2–3 (August 2009): 67–91. \url{https://doi.org/10.1007/s00799-009-0054-4}.

Olson, Jack E. \emph{Data Quality: The Accuracy Dimension}. Morgan Kaufmann, 2003. \url{https://books.google.com/books/about/Data_Quality.html?id=x8ahL57VOtcC}.

Palavitsinis, Nikos. “Metadata Quality Issues in Learning Repositories.” Alcala de Henares, 2014. \url{https://www.researchgate.net/publication/260424499_Metadata_Quality_Issues_in_Learning_Repositories}.

Palavitsinis, Nikos, Nikos Manouselis, and Salvador Sanchez-Alonso. “Metadata and Quality in Digital Repositories and Libraries from 1995 to 2015: A Literature Analysis and Classification.” \emph{International Information \& Library Review}, June 2, 2017, 11. \url{http://dx.doi.org/10.1080/10572317.2016.1278194}.

———. “Metadata Quality in Digital Repositories: Empirical Results from the Cross-Domain Transfer of a Quality Assurance Process: Journal of the American Society for Information Science and Technology.” \emph{Journal of the Association for Information Science and Technology} 65, no. 6 (June 2014): 1202–16. \url{https://doi.org/10.1002/asi.23045}.

———. “Metadata Quality in Learning Object Repositories: A Case Study.” \emph{The Electronic Library} 32, no. 1 (2014): 62–82. \url{https://doi.org/10.1108/EL-12-2011-0175}.

Park, Eun G. “Building Interoperable Canadian Architecture Collections: Initial Metadata Assessment.” \emph{The Electronic Library} 25, no. 2 (2007): 207–18. \url{https://doi.org/10.1108/02640470710741331}.

Park, Eun G., and Marc Richard. “Metadata Assessment in E-theses and Dissertations of Canadian Institutional Repositories.” \emph{The Electronic Library} 29, no. 3 (2011): 394–407. \url{https://doi.org/10.1108/02640471111141124}.

Park, Jung-ran. “Metadata Quality in Digital Repositories: A Survey of the Current State of the Art.” \emph{Cataloging \& Classification Quarterly} 47, no. 3–4 (2009): 213–28. \url{https://doi.org/10.1080/01639370902737240}.

———. “Semantic Interoperability and Metadata Quality: An Analysis of Metadata Item Records of Digital Image Collections.” \emph{Knowledge Organization} 33, no. 1 (2006): 20–34.

Park, Jung-ran, and Tosaka Yuji. “Metadata Quality Control in Digital Repositories and Collections: Criteria, Semantics, and Mechanisms.” \emph{Cataloging \& Classification Quarterly} 48, no. 8 (2010): 696–715. \url{https://doi.org/10.1080/01639374.2010.508711}.

Phillips, Mark. “Metadata Quality, Completeness, and Minimally Viable RecordsMetadata Quality, Completeness, and Minimally Viable Records.” Personal website. \emph{Mark E. Phillips Journal} (blog), January 5, 2015. \url{http://vphill.com/journal/post/4075/}.

Phipps, Jon, Diane I. Hillmann, and Gordon Paynter. “Orchestrating Metadata Enhancement Services: Introducing Lenny.” In \emph{Proceedings from the International Conference on Dublin Core and Metadata Applications, 2005}, 49–58. Madrid, Spain, 2005. \url{http://dcpapers.dublincore.org/pubs/article/view/803}.

Pirmann, Carrie. “Alternative Subject Languages for Cataloging,” Spring 2009. \url{http://courseweb.lis.illinois.edu/~pirmann2/LIS577/toolbox/langhead.html}.

Radulovic, Filip, Nandana Mihindukulasooriya, Raúl García-Castro, and Asunción Gómez-Pérez. “A Comprehensive Quality Model for Linked Data.” \emph{Semantic Web} Preprint, no. Preprint (2017): 1–22. \url{https://doi.org/10.3233/SW-170267}.

Rasaiah, Barbara A., Simon. D. Jones, Chris Bellman, Tim J. Malthus, and Andreas Hueni. “Assessing Field Spectroscopy Metadata Quality.” \emph{Remote Sensing} 7, no. 4 (2015): 4499–4526. \url{https://doi.org/10/gfphbq}.

Reiche, Konrad, and Edzard Höfig. “Implementation of Metadata Quality Metrics and Application on Public Government Data,” 236–41. Kyoto, Japan: Institute of Electrical and Electronics Engineers (IEEE), 2013. \url{https://doi.org/10.1109/COMPSACW.2013.32}.

Reiche, Konrad, Edzard Höfig, and Ina Schieferdecker. “Assessment and Visualization of Metadata Quality for Open Government Data.” In \emph{CeDEM14. Conference for E-Democracy and Open Governement}, 335–46. Krems, Austria: Edition Donau-Universität Krems, 2014. \url{http://www.donau-uni.ac.at/imperia/md/content/department/gpa/zeg/bilder/cedem/cedem14/cedem14_proceedings_1st_edition.pdf}.

Reiche, Konrad Johannes. “Assessment and Visualization of Metadata Quality for Open Government Data.” Master Thesis, Freie Universität Berlin, 2013. \url{http://www.inf.fu-berlin.de/inst/ag-se/theses/Reiche13-metadata-quality.pdf}.

Rennau, Hans-Jürgen. “Location Trees Enable XSD Based Tool Development.” In \emph{XML London 2017 Conference Proceedings}, 20–37. London, United Kingdom: XML London, 2017. \url{https://doi.org/10.14337/XMLLondon17.Rennau01}.

Rizza, Ettore, Anne Chardonnens, and Seth van Hooland. “Close-Reading of Linked Data: A Case Study in Regards to the Quality of Online Authority Files.” \emph{ArXiv E-Prints}, 2019. \url{https://arxiv.org/abs/1902.02140}.

Robertson, John R. “Metadata Quality: Implications for Library and Information Science Professionals.” \emph{Library Review} 54, no. 5 (2005): 295–300. \url{https://doi.org/10.1108/00242530510600543}.

Rula, Anisa, and Amrapali Zaveri. “Methodology for Assessment of Linked Data Quality.” In \emph{Proceedings of the 1st Workshop on Linked Data Quality}, edited by Magnus Knuth, Dimitris Kontokostas, and Harald Sack, Vol. 1215. CEUR Workshop Proceedings. CEUR, 2014. \url{http://ceur-ws.org/Vol-1215/paper-04.pdf}.

Shreeves, Sarah L., Joanne S. Kaczmarek, and Timothy W. Cole. “Harvesting Cultural Heritage Metadata Using the OAI Protocol.” \emph{Library Hi Tech} 21, no. 2 (2003): 159–69. \url{https://doi.org/10.1108/07378830310479802}.

Shreeves, Sarah L., Ellen M. Knutson, Besiki Stvilia, Carole L. Palmer, Michael B. Twidale, and Timothy W. Cole. “Is ‘Quality’ Metadata ‘Shareable’ Metadata? The Implications of Local Metadata Practices for Federated Collections.” In \emph{Currents and Convergence: Navigating the Rivers of Change. ACRL Twelfth National Conference.}, 223–37. Minneapolis, Minnesota: Association of College \& Research Libraries, 2005. \url{https://www.ideals.illinois.edu/bitstream/handle/2142/145/shreeves05.pdf}.

Sicilia, Miguel A., Elena Garcia, Carmen Pages, Jose J. Martinez, and Jose M. Gutierrez. “Complete Metadata Records in Learning Object Repositories: Some Evidence and Requirements.” \emph{International Journal of Learning Technology} 1, no. 4 (May 2005): 411–24. \url{https://doi.org/10.1504/IJLT.2005.007152}.

Simon, Agnès, Daniel Vila Suero, Eero Hyvönen, Esther Guggenheim, Lars G. Svensson, Nuno Freire, Rainer Simon, et al. “EuropeanaTech Task Force on a Multilingual and Semantic Enrichment Strategy: Final Report.” Task Force Report. Europeana, April 7, 2014. \url{http://pro.europeana.eu/get-involved/europeana-tech/europeanatech-task-forces/multilingual-and-semantic-enrichment-strategy}.

Simons, Gary, ed. “Open Language Archives Community (OLAC) Metadata Metrics,” June 29, 2009. \url{http://www.language-archives.org/NOTE/metrics.html}.

Snow, Karen. “Defining, Assessing, and Rethinking Quality Cataloging.” \emph{Cataloging \& Classification Quarterly}, 2017, 1–18. \url{https://doi.org/10.1080/01639374.2017.1350774}.

Soto-Hernández, Silvano, and Catalina Naumis-Peña. “Metadata in Mexican Television News Broadcasts on the Web.” In \emph{Joint Proceedings of the 1st Workshop on Temporal Dynamics in Digital Libraries (TDDL 2017), the (Meta)-Data Quality Workshop (MDQual 2017) and the Workshop on Modeling Societal Future (Futurity 2017) (TDDL\_MDQual\_Futurity 2017) Co-Located with 21st International Conference on Theory and Practice of Digital Libraries (TPLD 2017), Grand Hotel Palace, Thessaloniki, Greece, 21 September 2017}, Vol. 2038. CEUR Workshop Proceedings. CEUR Workshop Proceedings, n.d. \url{http://ceur-ws.org/Vol-2038/paper5.pdf}.

Stiller, Juliane, and Péter Király. “Multilinguality of Metadata. Measuring the Multilingual Degree of Europeana’s Metadata.” In \emph{Everything Changes, Everything Stays the Same? Understanding Information Spaces. Proceedings of the 15th International Symposium of Information Science (ISI 2017)}, edited by Maria Gäde, Violeta Trkulja, and Vivien Petras, 164–76. Schriften Zur Informationswissenschaft. Glückstadt, Germany: Verlag Werner Hülsbusch, 2017. \url{https://www.researchgate.net/publication/314879735_Multilinguality_of_Metadata_Measuring_the_Multilingual_Degree_of_Europeana%27s_Metadata}.

Stvilia, Besiki, and Les Gasser. “Value-Based Metadata Quality Assessment.” \emph{Library \& Information Science Research} 30, no. 1 (March 2008): 67–74. \url{https://doi.org/10.1016/j.lisr.2007.06.006}.

Stvilia, Besiki, Les Gasser, Michael B. Twidale, Sarah L. Shreeves, and Tim W. Cole. “Metadata Quality for Federated Collections.” In \emph{Proceedings of the Ninth International Conference on Information Quality (ICIQ-04)}, 111–25, 2004. \url{http://citeseerx.ist.psu.edu/viewdoc/download?doi=10.1.1.552.1921\&rep=rep1\&type=pdf}.

Stvilia, Besiki, Les Gasser, Michael B. Twidale, and Linda C. Smith. “A Framework for Information Quality Assessment.” \emph{Journal of the American Society for Information Science and Technology} 58, no. 12 (2007): 1720–33. \url{https://doi.org/10.1002/asi.20652}.

Stvilia, Besiki, C. Hinnant, S. Wu, A. Worrall, D. J. Lee, K. Burnett, G. Burnett, M. M. Kazmer, and P. F. Marty. “Research Project Tasks, Data, and Perceptions of Data Quality in a Condensed Matter Physics Community.” \emph{Journal of the Association for Information Science and Technology} 66, no. 2 (2015): 246–63. \url{https://doi.org/10.1002/asi.23177}.

Suominen, Osma, and Eero Hyvönen. “Improving the Quality of SKOS Vocabularies with Skosify.” In \emph{Knowledge Engineering and Knowledge Management: 18th International Conference, EKAW 2012, Galway City, Ireland, October 8-12, 2012.}, edited by Annette ten Teije, Johanna Völker, Siegfried Handschuh, Heiner Stuckenschmidt, Mathieu d’Acquin, Andriy Nikolov, Nathalie Aussenac-Gilles, and Nathalie Hernandez, 7603:383–97. Lecture Notes in Computer Science. Heidelberg: Springer, 2012. \url{https://doi.org/10.1007/978-3-642-33876-2_34}.

Suominen, Osma, and Christian Mader. “Assessing and Improving the Quality of SKOS Vocabularies.” \emph{Journal on Data Semantics} 3, no. 1 (2014): 47–73. \url{https://doi.org/10.1007/s13740-013-0026-0}.

Szostak, Rick, Andrea Scharnhorst, Wouter Beek, and Richard P. Smiraglia. “Connecting KOSs and the LOD Cloud.” \emph{ArXiv:1802.08141 [Cs]}, February 22, 2018. \url{http://arxiv.org/abs/1802.08141}.

Tallerås, Kim. “Quality of Linked Bibliographic Data: The Models, Vocabularies, and Links of Data Sets Published by Four National Libraries.” \emph{Journal of Library Metadata} 17, no. 2 (2017): 126–55. \url{https://doi.org/10.1080/19386389.2017.1355166}.

Tallerås, Kim, Jørn Helge B. Dahl, and Nils Pharo. “User Conceptualizations of Derivative Relationships in the Bibliographic Universe.” \emph{Journal of Documentation} 74, no. 4 (2018): 894–916. \url{https://doi.org/10.1108/JD-10-2017-0139}.

Tallerås, Kim, David Massey, Jørn Helge B. Dahl, and Nils Pharo. “Ordo Ad Chaos – Linking Norwegian Black Metal.” In \emph{Libraries, Black Metal and Corporate Finance: Current Research in Nordic Library and Information Science}, edited by Skans Kersti Nilsson and Anders Frenander, 136–50. Borås: University of Borås, 2013. \url{https://www.diva-portal.org/smash/get/diva2:883968/FULLTEXT01.pdf}.

Tani, Alice, Leonardo Candela, and Donatella Castelli. “Dealing with Metadata Quality: The Legacy of Digital Library Efforts.” \emph{Information Processing \& Management} 49, no. 6 (November 2013): 1194–1205. \url{https://doi.org/10.1016/j.ipm.2013.05.003}.

Tarver, Hannah, Mark Phillips, Oksana Zavalina, and Priya Kizhakkethil. “An Exploratory Analysis of Subject Metadata in the Digital Public Library of America.” In \emph{Proceedings from the International Conference on Dublin Core and Metadata Applications 2015}, 30–40. Sao Paolo, Brazil: Dublin Core Metadata Initiative, 2015. \url{https://digital.library.unt.edu/ark:/67531/metadc725779/}.

Tarver, Hannah, Oksana Zavalina, Mark Phillips, Daniel Gelaw Alemneh, and Shadi Shakeri. “How Descriptive Metadata Changes in the UNT Libraries’ Collection: A Case Study.” In \emph{Proceedings of the International Conference on Dublin Core and Metadata Applications}, 43–52. Austin, Texas, USA: Dublin Core Metadata Initiative, 2014. \url{http://dcevents.dublincore.org/IntConf/dc-2014/paper/view/235}.

Tcholtchev, Nikolay. “Visualization of Metadata Quality for Open Government Data.” Krems, Austria, 2014. \url{http://www.slideshare.net/dgpazegovzpi/konrad-cedem-praesi}.

Thomas, Sarah E. “Quality in Bibliographic Control.” \emph{Library Trends} 44, no. no.3 (winter 1996) (1996): 491–505.

Tönnies, Sascha. “Quality Control Using Semantic Technologies in Digital Libraries.” Doktors der Naturwissenschaften (Dr. rer. nat.), Technischen Universität Carolo-Wilhelmina zu Braunschweig, 2012. \url{https://publikationsserver.tu-braunschweig.de/receive/dbbs_mods_00046510}.

Tönnies, Sascha, and Wolf-Tilo Balke. “Quality Assessment in Digital Libraries - Challenges and Chances.” In \emph{Proceedings of the 22. GI-Workshop on Foundations of Databases (Grundlagen von Datenbanken)}, Vol. Vol-581. CEUR Workshop Proceedings, 2010. \url{http://ceur-ws.org/Vol-581/gvd2010_6_3.pdf}.

———. “Using Semantic Technologies in Digital Libraries - A Roadmap to Quality Evaluation.” In \emph{Research and Advanced Technology for Digital Libraries. ECDL 2009.}, edited by Maristella Agosti, José Borbinha, Sarantos Kapidakis, Christos Papatheodorou, and Giannis Tsakonas, 5714:168–79. Lecture Notes in Computer Science. Heidelberg: Springer, 2009. \url{https://doi.org/10.1007/978-3-642-04346-8_18}.

Trippel, Thorsten, Daan Broeder, Matej Durco, and Oddrun Ohren. “Towards Automatic Quality Assessment of Component Metadata.” In \emph{Proceedings of LREC 2014}, 3851–56. Reykjavik, Iceland, 2014. \url{http://hdl.handle.net/11858/00-001M-0000-0024-3233-2}.

Tsiflidou, Effie, and Nikos Manouselis. “Tools and Techniques for Assessing Metadata Quality.” In \emph{Metadata and Semantics Research: Proceedings of the 7th Research Conference, MTSR 2013, Thessaloniki, Greece, November 19-22, 2013.}, edited by Emmanouel Garoufallou and Jane Greenberg, 99–110. Thessaloniki, Greece: Springer, 2013. \url{https://doi.org/10.1007/978-3-319-03437-9_11}.

Van Kleeck, David, Gerald Langford, Jimmie Lundgren, Hikaru Nakano, Allison Jai O’Dell, and Trey Shelton. “Managing Bibliographic Data Quality in a Consortial Academic Library: A Case Study.” \emph{Cataloging \& Classification Quarterly} 54, no. 7 (2016): 452–67. \url{https://doi.org/10.1080/01639374.2016.1210709}.

Vassilakaki, Evgenia, and Emmanouel Garoufallou. “Multilingual Digital Libraries: A Review of Issues in System-Centered and User-Centered Studies, Information Retrieval and User Behavior.” \emph{International Information \& Library Review} 45, no. 1–2 (2013): 3–19. \url{https://doi.org/10.1016/j.iilr.2013.07.002}.

Ward, Jewel Hope. “A Quantitative Analysis of Dublin Core Metadata Element Set (DCMES) Usage in Data Providers Registered with the Open Archives Initiative (OAI).” Master’s paper, School of Information and Library Science of the University of North Carolina at Chapel Hill, 2002. \url{http://ils.unc.edu/MSpapers/2816.pdf}.

Welsh, Anne. “The Rare Books Catalog and the Scholarly Database.” \emph{Cataloging \& Classification Quarterly} 54, no. 5–6 (2016): 317–37. \url{https://doi.org/10.1080/01639374.2016.1188433}.

Westbrook, R. Niccole, Dan Johnson, Karen Carter, and Angela Lockwood. “Metadata Clean Sweep: A Digital Library Audit Project.” \emph{D-Lib Magazine} 18, no. 5/6 (June 2012). \url{https://doi.org/10.1045/may2012-westbrook}.

Young, Jasmine Y., John D. Westbrook, Zukang Feng, Raul Sala, Ezra Peisach, Thomas J. Oldfield, Sanchayita Sen, et al. “OneDep: Unified WwPDB System for Deposition, Biocuration, and Validation of Macromolecular Structures in the PDB Archive.” \emph{Structure} 25, no. 3 (March 7, 2017): 536–45. \url{https://doi.org/10.1016/j.str.2017.01.004}.

Zavalina, Oksana L. “Complementarity in Subject Metadata in Large-Scale Digital Libraries: A Comparative Analysis.” \emph{Cataloging \& Classification Quarterly} 52, no. 1 (January 2014): 77–89. \url{https://doi.org/10.1080/01639374.2013.848316}.

Zavalina, Oksana L., Priya Kizhakkethil, Daniel Gelaw Alemneh, Mark E. Phillips, and Hannah Tarver. “Building a Framework of Metadata Change to Support Knowledge Management.” \emph{Journal of Information \& Knowledge Management} 14, no. 01 (2015): 1550005-1-1550005–16. \url{https://doi.org/10.1142/S0219649215500057}.

Zavalina, Oksana L., Priya Kizhakkethil, and Shadi Shakeri. “Metadata Change in Traditional Library Collections and Digital Repositories: Exploratory Comparative Analysis.” \emph{Proceedings of the Association for Information Science and Technology} 52, no. 1 (2015): 1–5. \url{https://doi.org/10.1002/pra2.2015.1450520100146}.

Zavalina, Oksana L., Shadi Shakeri, Priya Kizhakkethil, and Mark E. Phillips. “Uncovering Hidden Insights for Information Management: Examination and Modeling of Change in Digital Collection Metadata.” In \emph{Transforming Digital Worlds}, edited by Gobinda Chowdhury, Julie McLeod, Val Gillet, and Peter Willett, 10766:645–51. Lecture Notes in Computer Science. Springer International Publishing, 2018. \url{https://doi.org/10/gfphbt}.

Zavalina, Oksana L., Vyacheslav Zavalin, Shadi Shakeri, and Priya Kizhakkethil. “Developing an Empirically-Based Framework of Metadata Change and Exploring Relation between Metadata Change and Metadata Quality in MARC Library Metadata.” \emph{Procedia Computer Science} 99 (2016): 50–63. \url{https://doi.org/10.1016/j.procs.2016.09.100}.

Zaveri, Amrapali, Dimitris Kontokostas, Mohamed A. Sherif, Lorenz Bühmann, Mohamed Morsey, and Sören Auer. “User-Driven Quality Evaluation of DBpedia.” In \emph{Proceedings of the 9th International Conference on Semantic Systems}, 97–104. I-SEMANTICS ’13. ACM, 2013. \url{https://doi.org/10.1145/2506182.2506195}.

Zaveri, Amrapali, Anisa Rula, Andrea Maurino, Ricardo Pietrobon, Jens Lehmann, and Sören Auer. “Quality Assessment for Linked Data: A Survey.” \emph{Semantic Web} 7, no. 1 (2016): 63–93. \url{https://doi.org/10.3233/SW-150175}.

Zeitlyn, David, and Megan Beardmore-Herd. “Testing Google Scholar Bibliographic Data: Estimating Error Rates for Google Scholar Citation Parsing.” \emph{First Monday} 23, no. Number 11-5 November 2018 (2018). \url{https://doi.org/10.5210/fm.v23i11.8658}.

Zeng, Marcia Lei, Bhagirathi Subrahmanyam, and Gregory M. Shreve. “Metadata Quality Study for the National Science Digital Library (NSDL) Metadata Repository.” In \emph{Digital Libraries: International Collaboration and Cross-Fertilization}, 339–40. Lecture Notes in Computer Science. Berlin, Heidelberg: Springer, 2004. \url{https://doi.org/10.1007/978-3-540-30544-6_36}.

